% !TEX TS-program = pdflatex
% !TEX encoding = UTF-8 Unicode

\documentclass[11pt]{report} % use larger type; default would be 10pt

\usepackage[utf8]{inputenc} % set input encoding (not needed with XeLaTeX)

%%% Examples of Article customizations
% These packages are optional, depending whether you want the features they provide.
% See the LaTeX Companion or other references for full information.

%%% PAGE DIMENSIONS
\usepackage{geometry} % to change the page dimensions
\geometry{a4paper} % or letterpaper (US) or a5paper or....
% \geometry{margin=2in} % for example, change the margins to 2 inches all round
% \geometry{landscape} % set up the page for landscape
%   read geometry.pdf for detailed page layout information

\usepackage{graphicx} % support the \includegraphics command and options

% \usepackage[parfill]{parskip} % Activate to begin paragraphs with an empty line rather than an indent

%%% PACKAGES
\usepackage{booktabs} % for much better looking tables
\usepackage{array} % for better arrays (eg matrices) in maths
\usepackage{paralist} % very flexible & customisable lists (eg. itemize/itemize, etc.)
\usepackage{verbatim} % adds environment for commenting out blocks of text & for better verbatim
\usepackage{subfig} % make it possible to include more than one captioned figure/table in a single float
% These packages are all incorporated in the memoir class to one degree or another...

%%% HEADERS & FOOTERS
\usepackage{fancyhdr} % This should be set AFTER setting up the page geometry
\pagestyle{fancy} % options: empty , plain , fancy
\renewcommand{\headrulewidth}{0pt} % customise the layout...
\lhead{}\chead{Links}\rhead{}
\lfoot{}\cfoot{\thepage}\rfoot{}

%%% SECTION TITLE APPEARANCE
\usepackage{sectsty}
\allsectionsfont{\sffamily\mdseries\upshape} % (See the fntguide.pdf for font help)
% (This matches ConTeXt defaults)

%%% ToC (table of contents) APPEARANCE
\usepackage[nottoc,notlof,notlot]{tocbibind} % Put the bibliography in the ToC
\usepackage[titles,subfigure]{tocloft} % Alter the style of the Table of Contents
\renewcommand{\cftsecfont}{\rmfamily\mdseries\upshape}
\renewcommand{\cftsecpagefont}{\rmfamily\mdseries\upshape} % No bold!

%%% END Article customizations

%%% The "real" document content comes below...

\title{Software Engineering Project - Links}
\author{Abhijith Madhav, Arjun S Bharadwaj}
%\date{} % Activate to display a given date or no date (if empty),
         % otherwise the current date is printed 

\begin{document}
\maketitle
\section*{User Requirements}
This section provides the list of features that are planned to be supported in the course of this project.
\begin{itemize}
\item
	Create user accounts/signup.
\item
Links Management
\begin{itemize}
	\item
		Save links.
	\item
		Edit links.
	\item
		Delete links.
	\item
		Tag links.
	\item
		Annotate links.
	\item
		Classify/Organize links in folders.
\end{itemize}
\item
Group Management
\begin{itemize}
	\item
		Create user groups.
	\item
		Join group.
	\item
		Unjoin the group.
	\item
		Add members to the group (by group owner).
	\item
		Remove members from group (by group owner).
\end{itemize}
\item
Share Options
\begin{itemize}
	\item
		Via Email.
	\item
		Via Twitter.
	\item
		Via Facebook.
	\item
		Share links with groups.
\end{itemize}
\item
Suggestions
\begin{itemize}
	\item
		Suggest tags if same link was shared by others if the links privacy is public.
	\item
		Suggest public links belonging to the same tags.
\end{itemize}
\item
Expose APIs
\begin{itemize}
	\item
	Search based on links, annotations, tags at various granularity levels of ownership of the links.
	\item
	APIs to add, delete and edit the links.
	\item
	Android App that consumes the data from the provided APIs.
\end{itemize}
\item
Shorten the links - URL shortener.
\end{itemize}

\end{document}
