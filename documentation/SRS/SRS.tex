% !TEX TS-program = pdflatex
% !TEX encoding = UTF-8 Unicode

\documentclass[11pt]{report} % use larger type; default would be 10pt

\usepackage[utf8]{inputenc} % set input encoding (not needed with XeLaTeX)

%%% Examples of Article customizations
% These packages are optional, depending whether you want the features they provide.
% See the LaTeX Companion or other references for full information.

%%% PAGE DIMENSIONS
\usepackage{geometry} % to change the page dimensions
\geometry{a4paper} % or letterpaper (US) or a5paper or....
% \geometry{margin=2in} % for example, change the margins to 2 inches all round
% \geometry{landscape} % set up the page for landscape
%   read geometry.pdf for detailed page layout information

\usepackage{graphicx} % support the \includegraphics command and options

% \usepackage[parfill]{parskip} % Activate to begin paragraphs with an empty line rather than an indent

%%% PACKAGES
\usepackage{booktabs} % for much better looking tables
\usepackage{array} % for better arrays (eg matrices) in maths
\usepackage{paralist} % very flexible & customisable lists (eg. itemize/itemize, etc.)
\usepackage{verbatim} % adds environment for commenting out blocks of text & for better verbatim
\usepackage{subfig} % make it possible to include more than one captioned figure/table in a single float
% These packages are all incorporated in the memoir class to one degree or another...

%%% HEADERS & FOOTERS
\usepackage{fancyhdr} % This should be set AFTER setting up the page geometry
\pagestyle{fancy} % options: empty , plain , fancy
\renewcommand{\headrulewidth}{0pt} % customise the layout...
\lhead{Software Requirements Specification for Links}\chead{}\rhead{}
\lfoot{}\cfoot{\thepage}\rfoot{}

%%% SECTION TITLE APPEARANCE
\usepackage{sectsty}
\allsectionsfont{\sffamily\mdseries\upshape} % (See the fntguide.pdf for font help)
% (This matches ConTeXt defaults)

%%% ToC (table of contents) APPEARANCE
\usepackage[nottoc,notlof,notlot]{tocbibind} % Put the bibliography in the ToC
\usepackage[titles,subfigure]{tocloft} % Alter the style of the Table of Contents
\renewcommand{\cftsecfont}{\rmfamily\mdseries\upshape}
\renewcommand{\cftsecpagefont}{\rmfamily\mdseries\upshape} % No bold!
\renewcommand{\chaptername}{}

\usepackage{titlesec}
\usepackage{lipsum}
\titleformat{\chapter}
{\filcenter\normalfont\LARGE\bfseries}
{\chaptertitlename~\thechapter} {0.5em} {}

%%% END Article customizations

%%% The "real" document content comes below...

\title{Software Requirements Specification for “Links” \\ Version 1.0}
\author{Abhijith Madhav, Arjun S Bharadwaj}
%\date{} % Activate to display a given date or no date (if empty),
         % otherwise the current date is printed 

\begin{document}
\maketitle
{\center {\LARGE { \bf Revisions} \\ } }
. \\
\begin{tabular}{ | c | c | c | c | }
\hline            
  Version & Primary Author(s) & Description of Version  & Date Completed\\
  \hline  
  \hline  
  1.0 & Abhijith Madhav(MT2013002) & Initial version sent to Client & \date{February 5, 2014} \\
   & Arjun S Bharadwaj(MT2013026) & & \\
\hline  
\hline  
  2.0 & Abhijith Madhav(MT2013002) & Elaboration of product scope. Made & \date{February 7, 2014} \\ 
   & Arjun S Bharadwaj(MT2013026) & interoperability as an FR.& \\
\hline  
\end{tabular}
\tableofcontents
\chapter{Introduction}
This document provides the Software Requirement Specifications of the project. Specifically it provides the details on the functional and non functional aspects of the project.
\section{Document Purpose}
The document describes the software requirements of “Links”, version 1.0. The software requirements encapsulate the whole of the product.

\section{Product Scope}
“Links” is website offering bookmarking services with the specific requirement being that it should be deployable on a private server within an organization.

The purpose is to enable people in an organization to share bookmarks with each other.

The specific benefit delivered by “Links” is that sharing/publishing of bookmarks can be controlled at the required level of granularity.

“Links” is intended to be an extensible, modular system with an exposed API. Third parties must be able to write plugins and applications using this API.

It is envisioned to seemlessly fit into the “share” motto of the net community, be it through simple email or through popular social platforms like facebook and twitter. A trigger based action execution like that of ifttt.com augments this motto and is intended to be a part of “Links”.

\maketitle
\chapter{Overall description}
This section provides the overall description of the project. This includes the product perspective and the functionality.
\section{Product Perspective}
A bookmarking service is a centralized online service which enables users to add, annotate, edit and share bookmarks of web documents, the most popular being Delicious. “Scuttle” is one of the other well known open source alternative. “Links” is envisioned on similar lines with the specific objective that it is aimed at deployment within an organization.
\section{Product Functionality}
\begin{enumerate}
\item
Provide for user accounts/signup.
\item
Provide for website bookmark management.
\item
Provide for management of user groups for sharing bookmarks.
\item
Provide for options to share website bookmarks amongst users and groups.
\item
Provide options to push website bookmarks onto popular social media and via email.
\end{enumerate}

\maketitle
\chapter{Functional Requirements}
This section specifies the functional requirements of the project.
\section{Functional Requirements}
This section provides the functional requirements of the project.
\begin{enumerate}
	\item
		Login
		\begin{itemize}
			\item
				User logs into the system.
			\item
				\emph{Input:} User enters the login details and clicks on login.
			\item
				\emph{Output:} 
					\begin{itemize}
						\item
							If the validation of the user credentials is successful:
							\begin{itemize}
								\item
									The user is redirected to the Homepage.
							\end{itemize}


						\item
							If the validation of the login details is unsuccessful:
							\begin{itemize}
								\item
									An appropriate error message is shown to the user.
							\end{itemize}
					\end{itemize}
		\end{itemize}

	\item
		Signup
		\begin{itemize}
			\item
				User clicks on Signup.
			\item
				\emph{Input:} Signup details are asked.
			\item
				\emph{Output:} 
					\begin{itemize}
						\item	
							If validation of the details is successful:
							\begin{itemize}
								\item
									Account is created.
							\end{itemize}

						\item
							If validation of the details is unsuccessful:
							\begin{itemize}
								\item
									An appropriate error message is shown to the user.
							\end{itemize}
					\end{itemize}
		\end{itemize}

	\item
		Save Links
		\begin{itemize}
			\item
				User bookmarks the links.
			\item
				\emph{Input:} A Title, URI, Tags and Annotations are provided.
			\item
				\emph{Output:} 
					\begin{itemize}
						\item				
							After the link is saved:
							\begin{itemize}
								\item
									Acknowledgement for the action is provided.
									\item
									Suggest tags if same link was shared by others based on the visibility level of the user.
							\end{itemize}			
					\end{itemize}					
		\end{itemize}

	\item
		Edit  Links
		\begin{itemize}
			\item
				User edits the links.
			\item
				\emph{Input:} An existing link, modified data (title, tags, annotations).
			\item
				\emph{Output:} 
					\begin{itemize}
						\item	
							After the link is updated:
						\begin{itemize}
							\item
								Acknowledgement for the update is provided.
						\end{itemize}						
					\end{itemize}			
		\end{itemize}


	\item
		Delete   Links
		\begin{itemize}
			\item
				User deletes  the links.
			\item
				\emph{Input:} An existing link shared by the user.
			\item
				\emph{Output:} 
					\begin{itemize}
						\item	
							After the link is deleted:
						\begin{itemize}
							\item
								Acknowledgement for the deletion is provided.
						\end{itemize}			
					\end{itemize}
			\end{itemize}

	\item
		Create User Groups
		\begin{itemize}
			\item
				Creation of Groups by the user.
			\item
				\emph{Input:} Group name and group description corresponding to the group.
			\item
				\emph{Output:} 
					\begin{itemize}
						\item
							If the group name is not already used:
						\begin{itemize}
							\item
								Acknowledgement for the creation is provided.
						\end{itemize}

						\item
							If the group name already exists:
						\begin{itemize}
							\item
								Appropriate error message.
						\end{itemize}
					\end{itemize}			
			\end{itemize}

	\item
		Join Group
		\begin{itemize}
			\item
				User joins the group.
			\item
				\emph{Input:} Group name of the group that the user wishes to subscribe to.
			\item
				\emph{Output:} Acknowledgement of the subscription is provided.
		\end{itemize}

	\item
		Unjoin Group
		\begin{itemize}
			\item
				User unjoins the group.
			\item
				\emph{Input:} Group name of the group that the user wishes to unsubscribe to.
			\item
				\emph{Output:} Acknowledgement of the un-subscription is provided.
		\end{itemize}

	\item
		Share
		\begin{itemize}
			\item
				Sharing of the links.
			\item
				\emph{Input:} A set of bookmarks to be shared via email or popular social media
			\item
				\emph{Output:} Share bookmarks via the specified media.
		\end{itemize}

	\item
		Add Members
		\begin{itemize}
			\item
				Add members to the group.
			\item
				\emph{Input:} User names by the group owner.
			\item
				\emph{Output:} Acknowledgement of addition to group.
		\end{itemize}


	\item
		Remove Members
		\begin{itemize}
			\item
				Remove members to the group.
			\item
				\emph{Input:} User names by the group owner.
			\item
				\emph{Output:} Acknowledgement of deletion to group.
		\end{itemize}
\item
Interoperability\\
Expose APIs through which the following can be done
\begin{itemize}
\item
Search based on bookmarks, annotations, tags in accordance to the visibility of the user.
\item
Add, delete and edit the bookmarks in accordance to the visibility of the user.
\end{itemize}
As a proof of concept, an android application that consumes the data from the provided APIs will be created
\end{enumerate}
\maketitle
\chapter{Non-functional Requirements}
This section specifies the non-functional aspects of the project.
\section{Performance Requirements}
\begin{enumerate}
\item
Any interactive transaction with “Links” should not take more than 5 seconds
\end{enumerate}
\section{Safety and Security Requirements}
\begin{enumerate}
\item
All users must be authenticated before being given modification privileges in “Links”.
\end{enumerate}
\end{document}
